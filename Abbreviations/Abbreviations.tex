\newglossaryentry{AVER}{
	sort=AVER,
	name= $\langle \quad \rangle$,
	description = symbol to indicate averaging,
	symbol =
}
\newglossaryentry{CONV}{
	sort=CONV,
	name= $*$,
	description = convolution operator,
	symbol =
}
\newglossaryentry{FT}{
	sort=FT,
	name=$FT$,
	description=Abbreviation for: Fourier transformation,
	symbol = 
	}
\newglossaryentry{FTs}{
	sort=FTs,
	name=$\mathcal{F_S}$,
	description=Symbol for sine FT,
	symbol = 
	}
\newglossaryentry{symbFT}{
	sort=FT,
	name=$\mathcal{F}$,
	description=Symbol for FT,
	symbol = 
	}
\newglossaryentry{symbFTinv}{
	sort=FTinv,
	name=$\mathcal{F}^{-1}$,
	description=Symbol for inverse FT,
	symbol = 
	}
\newglossaryentry{fQ}{
	sort=fQ,
	name=$f(Q)$,
	description= atomic form factor of atom i; it is dimensionless, as it refers to the Thompson scattering of an atom divided by the Thompson scattering of a single free electron,
	symbol = 
}
\newglossaryentry{FQ}{
	sort=FQPDF,
	name=$F(Q)$,
	description={The reduced an normailsed structure function which is transformed into G(R) by sine-FT. Unfortunately it has he same notation as the crystallographic structure factor, $F(\textbf{Q})$},
	symbol = 
	}
\newglossaryentry{FQStruc}{
	sort=FQ,
	name=$F(Q)$,
	description={The crystallographic structure factor. It is a complex quantity and its modulus is called the structure or scattering amplitude and the expression $|F(\textbf{Q})|^2$ gives the intensity $I(\textbf{Q})$},
	symbol = 
	}
\newglossaryentry{GR}{
	sort=GR,
	name=$G(r)$,
	description = {PDF, pair distribution function; reduced radial distribution function. $G(r)$ can be seen as a derivative of $R(r)$},
	symbol = [\AA\textsuperscript{-2}]
	}
\newglossaryentry{IQ}{
	sort=IQ,
	name=$I(Q)$,
	description = intensity of the scattered radiation at $\textbf{Q}$ or $Q$,
	symbol =
}

\newglossaryentry{KI}{
	sort=KI,
	name= {$\textbf{k}_i$},
	description={the wave vector which describes the direction and properties of the incident photons},
	symbol = [\AA\textsuperscript{-1}],
}
\newglossaryentry{KO}{
	sort=KO,
	name= {$\textbf{k}_o$},
	description={the wave vector which describes the direction and properties of the scattered photons},
	symbol = [\AA\textsuperscript{-1}],
}
\newglossaryentry{LAMBDA}{%
	name=$\lambda$,
	description={wavelenght of the applied radiation},
	symbol = [\AA{}],
	sort=LAMBDA,
}
\newglossaryentry{px}{
	sort=px,
	name=$p(\textbf{x})$,
	description = total distribution of atoms throughout the sample, "atom density",
	symbol = [\AA\textsuperscript{-1}]
}
\newglossaryentry{PR}{
	sort=PR,
	name=$P(\textbf{r})$ or $P(r)$,
	description= density-density or auto-correlation function; it is also called (3D)Patterson function; it contains all interatomic vectors $\textbf{r}$ or $r$ but they are shifted to the origin of the Patterson map,
	%The vector $\textbf{r}$ ia the magnitude that describes the projected interatomic vectors
	symbol = [\AA\textsuperscript{-1}]
}

\newglossaryentry{QVEC}{
	sort=QVEC,
	name= {$\textbf{Q}$},
	description={Scattering vector; defined by the relation $\textbf{Q} = 2 \pi (\textbf{K}_o - \textbf{K}_i)$},
	symbol = [\AA\textsuperscript{-1}],
}
\newglossaryentry{QVECSCAL}{
	sort=QVECSCAL,
	name={$Q$ or $|\textbf{Q}|$},
	description={Modulus of the scattering vector $\textbf{Q}$},
	symbol = [\AA\textsuperscript{-1}],
}
\newglossaryentry{QMAX}{
	sort=QMAX,
	name= $Q_{max}$,
	description= upper limit of the FT of the diffractogram to $G(r)$,
	symbol = [\AA\textsuperscript{-1}]
}
\newglossaryentry{QMIN}{
	sort=QMIN,
	name= $Q_{min}$,
	description= lower limit of the FT of the diffractogram to $G(r)$,
	symbol = [\AA\textsuperscript{-1}]
}
\newglossaryentry{QDAMP}{
	sort=QDAMP,
	name= $Q_{damp}$,
	description= parameter asserted to account for the influences of the limited experimental resolution on the PDF,
	symbol = [$r^{-1}$]
}
\newglossaryentry{QBROAD}{
	sort=QBROAD,
	name= $Q_{broad}$,
	description= parameter asserted to account for the influences of the limited experimental resolution on the PDF,
	symbol = [$r^{-1}$]
}

%\label{eq:RofrCalc}
%\newglossaryentry{RrCALC}{%
%	sort=RrCALC,
%	name=$R(\textbf{r})$,
%	description = {radial distribution function},
%	symbol = [distances.\AA\textsuperscript{-3}]
%}
%\newglossaryentry{RVEC}{
%	sort=RVEC,
%	name=$\textbf{r}$,
%	description = {radial distributional parameter; in principle it can be interpreted as the magnitude of the Patterson vector $\textbf{t}$},
%	symbol = [\AA{}]
%}
%\newglossaryentry{RVECSCAL}{
%	sort=RVECSCAL,
%	name={$r$ or $|\textbf{r}|$},
%	description = {modulus of the vector $\textbf{R}$},
%	symbol = [\AA{}]
%}
\newglossaryentry{Rij}{
	sort=Rij,
	name={$\textbf{r}_{ij}$ or $r_{ij}$},
	description = {distance between two atoms $\textbf{x}_j - \textbf{x}_i$; it is also the Patterson vector; its magnitude is interpreted as a radius $r$},
	symbol = [\AA{}]
}
\newglossaryentry{RHO0}{%
	name=$\rho_0$,
	description = {averaged auto-correlation; in literature tis quantity is also denoted be the term \textit{atom number density} which defines the mean number of atoms per unit volume at large values of $R$},
	symbol = [distances.\AA\textsuperscript{-3}],
	sort=RHO0,
}

\newglossaryentry{RHOR}{%
	name=$\rho(r)$,
	description = {density-density correlation function C(R) from which the self reference of each atom is subtracted},
	symbol = [distances.\AA\textsuperscript{-3}],
	sort=RHOR,
}
\newglossaryentry{SQ}{
	sort=SQ,
	name=$S(Q)$,
	description=reduced total scattering structure function (in some literature given in terms of the classical scattering of one electron),
	symbol = [au]
}
\newglossaryentry{SR}{
	sort=SR,
	name=$S(r)$,
	description=Patterson self-correlation,
	symbol = [au]
%	symbol = [distances.\AA\textsuperscript{-3}]
}
\newglossaryentry{THETA}{
	sort=THETA,
	name=$\theta$,
	description= {scattering angle; the theoretically possible range is from 0 to 180}; due to the experimental setup this can never be reached,
	symbol = [rad]
}
\newglossaryentry{Ti}{
	sort=Ti,
	name= $T_{i}(\textbf{Q})$,
	description=Debye-Waller factor of atom $i$ (see section \ref{sec:genExpScat}),
	symbol =
}

\newglossaryentry{WRrec}{
	sort=WRrec,
	name=$W(r)$,
	description=rectangle function the PDF is multiplied with in order to account for the measurement range (section~\ref{sec:measRange1}),
	symbol =
}
\newglossaryentry{X}{
	sort=X,
	name={$\textbf{x}_i$},
	description = {position vector of atom $i$},
	symbol = [\AA{}]
}
