% !TEX encoding =UTF-8 Unicode
% !TeX spellcheck = en_US
%\RequirePackage[l2tabu, orthodox]{nag}

%	\documentclass[12pt,twoside,onecolumn]{book} %draft im Entwurfsmodus
	%\documentclass[12pt,twoside,onecolumn,a4paper,draft]{book} %draft im Entwurfsmodus
	\usepackage[a4paper]{geometry}
%	\frenchspacing 				%Keine längeren Abstände nach Punkten.			
	\raggedbottom
%	\flushbottom		%------ Das streckt den Text und füllt die Seiten. Na ja!


%------ Sprachoptionen und Schriftoptionen
  % --- Latex	
	\usepackage[british]{babel}
	\usepackage[utf8]{inputenc}
	\usepackage[T1]{fontenc}
	\usepackage{kurier}
%	\usepackage{quattrocento}
%	\usepackage{avant}			%renew
%	\usepackage{bookman}
%	\usepackage{tgadventor}		%renew
%	\usepackage{emerald}
%	\renewcommand*\familydefault{\sfdefault}			%standard serifenlos
%	\renewcommand*\familydefault{\ttdefault}			%standard typewriter
	
	\usepackage{upgreek}
	\usepackage{units}	
	\usepackage{textcomp}
	\usepackage{microtype}
	\usepackage{url}
	
	\usepackage[hidelinks]{hyperref}
	
	\usepackage{setspace}
	\onehalfspacing
	
	\usepackage{tcolorbox}
	\usepackage[export]{adjustbox}
	
	%------ Mathematik
	\usepackage{amsmath,amssymb,amsthm,amstext,amsfonts,mathrsfs}
%		\numberwithin{equation}{chapter}
	
	\usepackage{nccmath}		%Für linksbündige Formeln
	\usepackage{esvect}			%Vektorenpfeile
	%\usepackage{titlesec}
	
	%------ Für Tabellen
%	\usepackage[table]{xcolor}
	\usepackage{tabularx}
	\usepackage{array}
	\usepackage{multirow}
	\usepackage{booktabs}
	\usepackage{multicol}
	\usepackage{lscape}
	\usepackage{rotating}
	
	
	%------ Für Graphiken
	\usepackage{graphicx}		% Paket zur Verwendung von .jpg Bildern
	%\usepackage{subfig}
	\usepackage{subcaption}
	\usepackage{caption}[2013/02/03]	% Beschriftungen der Graphiken in Minipage zentriert
	\captionsetup{margin=10pt,font={small,bf},labelfont=bf,justification=centering}
	\usepackage[section]{placeins}	%Bilder bleiben im jeweiligen Abschnitt
	\usepackage{wrapfig}			%Textumflossene Graphik
	%	\usepackage{picins}
	%	\usepackage{sidecap}			%Seitliche Beschriftungen - gibt besseres denke ich
	
	%------ Für Zahl der Ebenen im Inhaltsverzeichnis
	\setcounter{secnumdepth}{2}
	\setcounter{tocdepth}{4}
	
	%------ Beeinflusst Nummerierungen der Elemente
%	\usepackage{chngcntr}
%	\counterwithout{footnote}{chapter}
%	\counterwithout{figure}{chapter}
%	\counterwithout{table}{chapter}
%	\counterwithout{equation}{chapter}
	
%	\renewcommand\theequation{\thechapter.\arabic{equation}

	%------ Hurenkinder und Schusterjungen verhindern
	\clubpenalty =03000
	\widowpenalty=30000
	%\displaywidowpenalty=3000
	
	%------ zB Tiefgestellung
%	\usepackage{fixltx2e} %Koma braucht es nicht
	
	\usepackage{pdfpages}
	
	
	\usepackage{comment}
	
	\usepackage{longtable}
	\usepackage{newclude}
	%\usepackage{epstopdf}
	\usepackage{textgreek}
	
	%\usepackage{pythontex}
	%\usepackage{fvextra}
	
	\usepackage{etoolbox}
	\usepackage{flafter}
	%\epstopdfDeclareGraphicsRule{.tif}{png}{.png}{convert #1 \OutputFile}
	%\AppendGraphicsExtensions{.tif}
	
	
	\usepackage{epigraph}
	\epigraphfontsize{\small\itshape}
	\setlength\epigraphwidth{12cm}
	\setlength\epigraphrule{0pt}
	
	\usepackage{minted}
	%\newmintinline[inline]{text}{fontsize=\normalsize}
	
	%\usepackage{listings}
	\usepackage{color}
	
	\usepackage[version=4]{mhchem}
	
	\definecolor{dkgreen}{rgb}{0,0.6,0}
	\definecolor{gray}{rgb}{0.5,0.5,0.5}
	\definecolor{mauve}{rgb}{0.58,0,0.82}
	
	%\lstset{frame=tb,
	%  language=Python,
	%  aboveskip=3mm,
	%  belowskip=3mm,
	%  showstringspaces=false,
	%  columns=flexible,
	%  basicstyle={\small\ttfamily},
	%  numbers=none,
	%  numberstyle=\tiny\color{gray},
	%  keywordstyle=\color{blue},
	%  commentstyle=\color{dkgreen},
	%  stringstyle=\color{mauve},
	%  breaklines=true,
	%  breakatwhitespace=true,
	%  tabsize=3,
	%  escapeinside=\#
	%}
	
	
	
	%\usepackage{etoolbox}% http://ctan.org/pkg/etoolbox
	%
	%\makeatletter
	%\patchcmd{\l@chapter}% <cmd>
	%{1.5em}% <search>
	%{5cm}% <replace>
	%{}{}% <success><failure>
	%\makeatother
	%
	%\makeatletter
	%\patchcmd{\@makechapterhead}% <cmd>
	%{\space}% <search>
	%{\hspace*{3cm}\ignorespaces}% <replace>
	%{}{}% <success><failure>
	%\makeatother
	
	
	
	\hyphenation{char-ac-ter-i-sa-tion}
	
	
	%------ selbstdefinierte Symbole
	\newcommand{\standb}			%------ Zeichen für Standardbedingungen
	{{\unitlength1ex\linethickness{0.1ex}%
			\begin{picture}(2,2)%
			\put(1,1){\circle{1}}%
			\put(0.25,1){\line(1,0){1.5}}%
			\end{picture}}}
	

