\usepackage{calc}
\newlength{\marginWidth}
\setlength\marginWidth{\marginparwidth+\marginparsep}
\newlength{\fulllinewidth}
\setlength\fulllinewidth{\textwidth+\marginWidth}

\usepackage{truncate} %Um zu lange Kapiteltitel abzuschneiden

\footskip=1.6cm
\makeatletter % = mache @ letter 

%Vordefinition mehrfachverwendeter Teile
\def\oddfootSTANDARD{
   \renewcommand{\@oddfoot}{
       \hbox to\textwidth{\vbox{\hbox to\textwidth{
          \hfill
          \strut
          \hspace{1pt}
       }}}
       \hbox to\marginWidth{\vbox{\hbox to\marginWidth{
          \strut %unsichtbares Zeichen
               \large
               \hspace{5pt}               
               \vrule width 1pt height 1cm
            \hspace{8pt}            
            \textsf{\thepage}
            \hfill
       }}}\hss   
   }
}

\def\evenfootSTANDARD{
   \renewcommand{\@evenfoot}{
      \hspace{-\marginWidth}  
         \hbox to\marginWidth{\vbox{\hbox to\marginWidth{
         \large
         \strut %unsichtbares Zeichen
         \hfill
         \textsf{\thepage}
         \hspace{5pt}
         \vrule width 1pt height 1cm
         \hspace{7pt}
      }}}\hss
   }  
}


%Standardstil für die gesamte Dissertation
\newcommand{\ps@thesisMAIN}{
   \renewcommand{\@oddhead}{
         \hbox to\textwidth{\vbox{\hbox to\textwidth{
            \textsf
            \hfill
            \rightmark
            \strut
            \hspace{1pt}
      }}}
         \hbox to\marginWidth{\vbox{\hbox to\marginWidth{
            \strut %unsichtbares Zeichen
            \hspace{5pt}
            \vrule width 1pt
            \hspace{5pt}
            \textsf
            \thesection
            \hfill
         }}}\hss
   }  
   
   \renewcommand{\@evenhead}{
      \hspace{-\marginWidth}  
         \hbox to\marginWidth{\vbox{\hbox to\marginWidth{
            \hfill
            \strut %unsichtbares Zeichen
            \textbf{\textsf{Chapter~\thechapter}}
            \hspace{5pt}
            \vrule width 1pt
            \hspace{7pt}
            \strut
         }}}\hss
         
         \hbox to\textwidth{\vbox{\hbox to\textwidth{
            \strut %unsichtbares Zeichen
         \truncate{.9\textwidth}{\leftmark}
         \hfill
      }}}\hss
   }	
   
   \oddfootSTANDARD   
   \evenfootSTANDARD   
}


%Der PLAIN-Style der Chapter- und Sonderseiten muss redefiniert werden.
\renewcommand{\ps@plain}{
   \let\@oddhead\@empty
   \let\@evenhead\@empty
   \let\@evenfoot\@empty   
   \oddfootSTANDARD
}

%Spezieller Stil für Inhaltsverzeichnis und Literaturverzeichnis (ohne Nummern wie 0.0 oder B.0)
\newcommand{\ps@thesisINTRO}{
   \renewcommand{\@oddhead}{
         \hbox to\textwidth{\vbox{\hbox to\textwidth{
            \textsf
            \hfill
            \sffamily\rightmark
            \strut
            \hspace{1pt}
         }}}\hss
   } 
   
   \renewcommand{\@evenhead}{
         \hbox to\textwidth{\vbox{\hbox to\textwidth{
            \strut %unsichtbares Zeichen
            \truncate{.9\textwidth}{\sffamily\leftmark}
            \hfill
         }}}\hss  
   }
   
   \oddfootSTANDARD   
   \evenfootSTANDARD   
}

%Spezieller Stil für Anhänge
%	\renewcommand{\chaptermark}[1]{\markboth{#1}{}}
%	\renewcommand{\sectionmark}[1]{\markright{\thesection\ #1}}

\newcommand{\ps@thesisBACKMATTER}{
   \renewcommand{\@oddhead}{
         \hbox to\textwidth{\vbox{\hbox to\textwidth{
            \textsf
            \hfill
            \rightmark
            \strut
            \hspace{1pt}
         }}}
         \hbox to\marginWidth{\vbox{\hbox to\marginWidth{%
            \strut %unsichtbares Zeichen
            \hspace{5pt}
            \vrule width 1pt
            \hspace{5pt}
            \textsf
            \textbf{\textsf{}}
            %\thesection
            \hfill
         }}}\hss
   }
   
   \renewcommand{\@evenhead}{
      \hspace{-\marginWidth}  
         \hbox to\marginWidth{\vbox{\hbox to\marginWidth{
            \hfill
            \strut %unsichtbares Zeichen
            \textbf{\textsf{}}
            \hspace{5pt}
            \vrule width 1pt
            \hspace{7pt}
            \strut
         }}}\hss
         
         \hbox to\textwidth{\vbox{\hbox to\textwidth{
            \strut %unsichtbares Zeichen
            \truncate{.9\textwidth}{\leftmark}
            \hfill
         }}}\hss  
   }
   
   \oddfootSTANDARD
   \evenfootSTANDARD
}


\newcommand{\ps@thesisAPPENDIX}{
   \renewcommand{\@oddhead}{
         \hbox to\textwidth{\vbox{\hbox to\textwidth{
            \textsf
            \hfill
            \rightmark
            \strut
            \hspace{1pt}
         }}}
         \hbox to\marginWidth{\vbox{\hbox to\marginWidth{%
            \strut %unsichtbares Zeichen
            \hspace{5pt}
            \vrule width 1pt
            \hspace{5pt}
            \textsf
            \textbf{\textsf{APPENDIX}}
            %\thesection
            \hfill
         }}}\hss
   }
   
   \renewcommand{\@evenhead}{
      \hspace{-\marginWidth}  
         \hbox to\marginWidth{\vbox{\hbox to\marginWidth{
            \hfill
            \strut %unsichtbares Zeichen
            %\textbf{\textsf{APPENDIX}}
            \hspace{5pt}
            \vrule width 1pt
            \hspace{7pt}
            \strut
         }}}\hss
         
         \hbox to\textwidth{\vbox{\hbox to\textwidth{
            \strut %unsichtbares Zeichen
            \truncate{.9\textwidth}{\leftmark}
            \hfill
         }}}\hss  
   }
   
   \oddfootSTANDARD
   \evenfootSTANDARD
}


\newcommand{\ps@thesisBIBLIOGRAPHY}{
   \renewcommand{\@oddhead}{
         \hbox to\textwidth{\vbox{\hbox to\textwidth{
            \textsf
            \hfill
            \rightmark
            \strut
            \hspace{1pt}
         }}}
         \hbox to\marginWidth{\vbox{\hbox to\marginWidth{%
            \strut %unsichtbares Zeichen
            \hspace{5pt}
            \vrule width 1pt
            \hspace{5pt}
            \textsf
            \textbf{\textsf{}}
            %\thesection
            \hfill
         }}}\hss
   }
   
   \renewcommand{\@evenhead}{
      \hspace{-\marginWidth}  
         \hbox to\marginWidth{\vbox{\hbox to\marginWidth{
            \hfill
            \strut %unsichtbares Zeichen
            \textbf{\textsf{}}
            \hspace{5pt}
            \vrule width 1pt
            \hspace{7pt}
            \strut
         }}}\hss
         
         \hbox to\textwidth{\vbox{\hbox to\textwidth{
            \strut %unsichtbares Zeichen
            \truncate{.9\textwidth}{\leftmark}
            \hfill
         }}}\hss  
   }
   
   \oddfootSTANDARD
   \evenfootSTANDARD
}



\newcommand{\ps@reallyempty}{
   \let\@oddhead\@empty
   \let\@evenhead\@empty
   \let\@oddfoot\@empty
   \let\@evenfoot\@empty
}

\renewcommand{\chaptermark}[1]{\markboth{{\textsf{#1}}}{}}%markboth hat zwei argumente für die linke und rechte seite -- zwischen {{ kann zB \uppercase gesetzt werden
\renewcommand{\sectionmark}[1]{\markright{\textsf{#1}}}

\makeatother % = mache @ wieder zu nicht-Buchstaben 
%\pagestyle{thesis}


		%Problem mit den Seitenzahlen und Headern auf leeren Seiten nach Kapiteln:
%		\let\origdoublepage\cleardoublepage
%		\newcommand{\clearemptydoublepage}{%
%		  \clearpage
%		  {\pagestyle{empty}\origdoublepage}%
%		}		
%		\let\cleardoublepage\clearemptydoublepage
%------------------------------------------------------------------------


	
	
